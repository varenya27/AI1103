\documentclass[journal,12pt,twocolumn]{IEEEtran}
\title{Assignment 1}
\author{Varenya Upadhyaya EP20BTECH11026}
\date{}

\usepackage[utf8]{inputenc}
\usepackage{amsmath}
\usepackage{enumitem}
\usepackage{multicol}
\usepackage[english]{babel}
\usepackage{listings}

\lstset{
%language=C,
frame=single, 
breaklines=true,
columns=fullflexible
}
\begin{document}

\maketitle
Download all python codes from: 
\begin{lstlisting}
https://github.com/varenya27/AI1103/blob/main/Assignment%201/Assignment_1.py
\end{lstlisting}
and latex-tikz codes from:
\begin{lstlisting}
https://github.com/varenya27/AI1103/blob/main/Assignment%201/main.tex
\end{lstlisting}
\maketitle   
\begin{center}
\section*{Problem 5.24}
\end{center}
One card is drawn from a well-shuffled deck of 52 cards. Calculate the probability that the card will:
\begin{enumerate}[label=(\roman*)]
    \item be an ace,
    \item not be an ace.
\end{enumerate}
\begin{center}
\maketitle
    \section*{Solution}
\end{center}
It is known that the total number of cards in the deck is 52, out of which there are four aces. Let random variable $X \in \{0,1\}$ denote the possible outcomes of the experiment of drawing a card from the shuffled deck.\\
\begin{center}
    \begin{tabular}{|c|c|c|}
    \hline
    \textbf{Card} & \textbf{X} & \textbf{Number}\\
    \hline
    Ace & 0 & n(X = 0) = 4\\
    \hline
    Not an Ace & 1 & n(X = 1) = 48\\
    \hline
    \end{tabular}
\end{center}


\begin{equation}
    p(X=0) = \frac{n(X = 0)}{n(X = 0) + n(X = 1)} = \frac{4}{52}\\
\end{equation}
\begin{equation}
    \Rightarrow p(X=0) = 0.076923\\
\end{equation}
Similarly,\\
\begin{equation}
    p(X = 1) = \frac{n(X=1)}{n(X=0)+n(X=1)} = \frac{48}{52}
\end{equation}
\begin{equation}
    \Rightarrow p(X=1) = 0.923077
\end{equation}
\\Hence, the required probabilities are:
\begin{enumerate}[label=(\roman*)]
    \item 0.076923
    \item 0.923077
\end{enumerate}
\end{document}

