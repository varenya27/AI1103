\documentclass[journal,12pt,twocolumn]{IEEEtran}
\title{Assignment 2}
\author{Varenya Upadhyaya EP20BTECH11026}
\date{}

\usepackage[utf8]{inputenc}
\usepackage{amsmath}
\usepackage{enumitem}
\usepackage{multicol}
\usepackage[english]{babel}
\usepackage{listings}
\usepackage{tabularx}
\usepackage{longtable}
\usepackage{graphicx}

\lstset{
%language=C,
frame=single, 
breaklines=true,
columns=fullflexible
}
\begin{document}

\maketitle
Download all latex-tikz codes from:
\begin{lstlisting}
https://github.com/varenya27/AI1103/blob/main/Assignment2/main.tex
\end{lstlisting}
\maketitle   
\begin{center}
\section*{\textbf{Problem}}
\end{center}
If P and Q are two random events, then the following is true:
\begin{enumerate}[label = (\alph*)]
    \item Independence of P and Q implies that probability $(P\cap Q)=0 $ 
    \item Probability $(P\cup Q) \leq   $ Probability $(P)$ + Probability(Q) 
    \item If P and Q are mutually exclusive, then they must be independent
    \item Probability $(P\cap Q) \leq$ Probability $(P)$
\end{enumerate}
\maketitle
\section*{\textbf{Solution}}
For two random events A and B that are independent, we know that, 
\begin{align}
Pr(A\cap B) = Pr(A)Pr(B)
\end{align}
and for two mutually exclusive events C and D, 
\begin{align}
    Pr(C\cap D) = 0
\end{align}

\begin{enumerate}[label = (\alph*)]
    \item Independence of P and Q implies that the occurrence of one is unaffected by the other. 
    \begin{align}
       \Rightarrow Pr(P\cap Q) = Pr(P)Pr(Q)
    \end{align}
    The given option will be true only when either Pr(P) or Pr(Q) will be zero, therefore, (a) is incorrect.\\
    \item From set theory,
    \begin{align}
    A\cup B &= A + B - A\cap B\\
    \Rightarrow Pr(P+Q) &= Pr(P) + Pr(Q) - Pr(P\cap Q)\\
    \Rightarrow Pr(P+Q) &\leq Pr(P) + Pr(Q)
    \end{align}
    thus, (b) is incorrect.\\
    \item Two events can be both mutually exclusive and independent only when one of them have a zero probability. Since it isn't necessary that $Pr(P)=0$ or $Pr(Q)=0$, (c) is incorrect.\\
    \item The set $P$ will have either have the same or more elements than the set $P\cap Q$
    \begin{equation}
        Pr(P\cap Q) \leq Pr(P)
    \end{equation}
    (d) is correct.\\
\end{enumerate}
Thus, the only correct option is (d).

\end{document}

