\documentclass[journal,12pt,twocolumn]{IEEEtran}

\usepackage{setspace}
\usepackage{gensymb}
\singlespacing
\usepackage[cmex10]{amsmath}

\usepackage{amsthm}

\usepackage{mathrsfs}
\usepackage{txfonts}
\usepackage{stfloats}
\usepackage{bm}
\usepackage{cite}
\usepackage{cases}
\usepackage{subfig}

\usepackage{longtable}
\usepackage{multirow}

\usepackage{enumitem}
\usepackage{mathtools}
\usepackage{steinmetz}
\usepackage{tikz}
\usepackage{circuitikz}
\usepackage{verbatim}
\usepackage{tfrupee}
\usepackage[breaklinks=true]{hyperref}
\usepackage{graphicx}
\usepackage{tkz-euclide}

\usetikzlibrary{calc,math}
\usepackage{listings}
    \usepackage{color}                                            %%
    \usepackage{array}                                            %%
    \usepackage{longtable}                                        %%
    \usepackage{calc}                                             %%
    \usepackage{multirow}                                         %%
    \usepackage{hhline}                                           %%
    \usepackage{ifthen}                                           %%
    \usepackage{lscape}     
\usepackage{multicol}
\usepackage{chngcntr}

\DeclareMathOperator*{\Res}{Res}
\renewcommand\thesection{\arabic{section}}
\renewcommand\thesubsection{\thesection.\arabic{subsection}}
\renewcommand\thesubsubsection{\thesubsection.\arabic{subsubsection}}

\renewcommand\thesectiondis{\arabic{section}}
\renewcommand\thesubsectiondis{\thesectiondis.\arabic{subsection}}
\renewcommand\thesubsubsectiondis{\thesubsectiondis.\arabic{subsubsection}}


\hyphenation{op-tical net-works semi-conduc-tor}
\def\inputGnumericTable{}                                 %%

\newtheorem{theorem}{Theorem}[section]
\newtheorem{problem}{Problem}
\newtheorem{proposition}{Proposition}[section]
\newtheorem{lemma}{Lemma}[section]
\newtheorem{corollary}[theorem]{Corollary}
\newtheorem{example}{Example}[section]
\newtheorem{definition}[problem]{Definition}

\newcommand{\BEQA}{\begin{eqnarray}}
\newcommand{\EEQA}{\end{eqnarray}}
\newcommand{\define}{\stackrel{\triangle}{=}}
\bibliographystyle{IEEEtran}
\raggedbottom
\setlength{\parindent}{0pt}
\providecommand{\mbf}{\mathbf}
\providecommand{\pr}[1]{\ensuremath{\Pr\left(#1\right)}}
\providecommand{\qfunc}[1]{\ensuremath{Q\left(#1\right)}}
\providecommand{\sbrak}[1]{\ensuremath{{}\left[#1\right]}}
\providecommand{\lsbrak}[1]{\ensuremath{{}\left[#1\right.}}
\providecommand{\rsbrak}[1]{\ensuremath{{}\left.#1\right]}}
\providecommand{\brak}[1]{\ensuremath{\left(#1\right)}}
\providecommand{\lbrak}[1]{\ensuremath{\left(#1\right.}}
\providecommand{\rbrak}[1]{\ensuremath{\left.#1\right)}}
\providecommand{\cbrak}[1]{\ensuremath{\left\{#1\right\}}}
\providecommand{\lcbrak}[1]{\ensuremath{\left\{#1\right.}}
\providecommand{\rcbrak}[1]{\ensuremath{\left.#1\right\}}}
\theoremstyle{remark}
\newtheorem{rem}{Remark}
\newcommand{\sgn}{\mathop{\mathrm{sgn}}}
% \providecommand{\abs}[1]{\left\vert#1\right\vert}
\providecommand{\res}[1]{\Res\displaylimits_{#1}} 
% \providecommand{\norm}[1]{\left\lVert#1\right\rVert}
%\providecommand{\norm}[1]{\lVert#1\rVert}
\providecommand{\mtx}[1]{\mathbf{#1}}
% \providecommand{\mean}[1]{E\left[ #1 \right]}
\providecommand{\fourier}{\overset{\mathcal{F}}{ \rightleftharpoons}}
%\providecommand{\hilbert}{\overset{\mathcal{H}}{ \rightleftharpoons}}
\providecommand{\system}{\overset{\mathcal{H}}{ \longleftrightarrow}}
	%\newcommand{\solution}[2]{\textbf{Solution:}{#1}}
\newcommand{\solution}{\noindent \textbf{Solution: }}
\newcommand{\cosec}{\,\text{cosec}\,}
\providecommand{\dec}[2]{\ensuremath{\overset{#1}{\underset{#2}{\gtrless}}}}
\newcommand{\myvec}[1]{\ensuremath{\begin{pmatrix}#1\end{pmatrix}}}
\newcommand{\mydet}[1]{\ensuremath{\begin{vmatrix}#1\end{vmatrix}}}
\numberwithin{equation}{subsection}
\makeatletter
\@addtoreset{figure}{problem}
\makeatother
\let\StandardTheFigure\thefigure
\let\vec\mathbf
\renewcommand{\thefigure}{\theproblem}
\def\putbox#1#2#3{\makebox[0in][l]{\makebox[#1][l]{}\raisebox{\baselineskip}[0in][0in]{\raisebox{#2}[0in][0in]{#3}}}}
     \def\rightbox#1{\makebox[0in][r]{#1}}
     \def\centbox#1{\makebox[0in]{#1}}
     \def\topbox#1{\raisebox{-\baselineskip}[0in][0in]{#1}}
     \def\midbox#1{\raisebox{-0.5\baselineskip}[0in][0in]{#1}}

\lstset{
%language=C,
frame=single, 
breaklines=true,
columns=fullflexible
}
\title{Assignment 5}
\author{Varenya Upadhyaya EP20BTECH11026}
\date{}
\begin{document}

\maketitle
% Download all python codes from:
% \begin{lstlisting}
% https://github.com/varenya27/AI1103/blob/main/Assignment5/codes
% \end{lstlisting}
Download all latex-tikz codes from:
\begin{lstlisting}
https://github.com/varenya27/AI1103/blob/main/Assignment5/main.tex
\end{lstlisting}
\maketitle   
\begin{center}
\section*{\textbf{Problem}}
\end{center}
Men arrive in a queue according to a Poisson process with rate $\lambda_1$ and women arrive in the same queue according to another Poisson process with rate $\lambda_2$. The arrivals of men and women are independent. The probability that the first arrival is a man is
\begin{enumerate}[label = (\alph*)]
\begin{multicols}{4}
\setlength\itemsep{2em}
    \item $\dfrac{\lambda_1}{\lambda_1+\lambda_2}$
    \item $\dfrac{\lambda_2}{\lambda_1+\lambda_2}$
    \item $\dfrac{\lambda_1}{\lambda_2}$
    \item $\dfrac{\lambda_2}{\lambda_1}$
\end{multicols}
\end{enumerate}

\maketitle
\section*{\textbf{Solution}}

Let $X$ and $Y$ be two discrete random variables that represent the number of men and women arriving in the queue in a given time interval, and $E$ be the event that a man arrives first. For a time interval of $t$ units: 
\begin{table}[h]
    \centering
    \resizebox{\columnwidth}{!}{
    \begin{tabular}{|l|l|l|l|}
        \hline     
        \textbf{Random variable}&\textbf{Event occurring}&\textbf{Rate}&\textbf{Poisson parameter}  \\\hline
        $X$&Men arriving in the queue&$\lambda_1$&$\lambda_1\times t$\\ \hline
        $Y$&Women arriving in the queue&$\lambda_2$&$\lambda_2\times t$\\ \hline
        $X+Y$&Any person arriving&$\lambda_3=\lambda_1+\lambda_2$& $\lambda_3\times t$\\ \hline
    \end{tabular}
    }
    \caption{Outcome of the Experiment}
    \label{tab:my_label}
\end{table}\\
For a Poisson distribution, with parameter $\lambda$, we know that,
\begin{align}
    \pr{X=k} = \frac{e^{-\lambda}{\lambda^{k}}}{k!}
\end{align}
Let $(0,t)$ be the time interval in which the first person arrives.
\begin{align}
    \pr{X+Y=1} &= \frac{e^{-\lambda_3 t}{(\lambda_3 t)^{1}}}{1!}\\
    &= e^{-\lambda_3 t}{\lambda_3 t}
\end{align}
The condition is that the first person arriving has to be a man,
\begin{align}
    \pr{X=1} &= \frac{e^{-\lambda_1 t}{(\lambda_1 t)^{1}}}{1!}\\
    &= e^{-\lambda_1 t}{\lambda_1 t}\\
    \pr{Y=0} &= \frac{e^{-\lambda_2 t}{(\lambda_2 t)^{0}}}{0!}\\
    &= e^{-\lambda_2 t}
\end{align}
The required probability can be calculated as,
\begin{align}
    \pr{E}&=\pr{(X=1)(Y=0)\;\vert{\;X+Y=1}} \\
    &=\frac{\pr{X=1}\times\pr{Y=0}}{\pr{X+Y=1}}\\
    &= \frac{{e^{-\lambda_1 t} \lambda_1 t}\times{e^{-\lambda_2 t}}}{e^{-\lambda_3 t}\times{\lambda_3 t}}\\
    &= \frac{e^{-(\lambda_1+\lambda_2)t}}{e^{-\lambda_3 t}}\times \frac{\lambda_1 t}{\lambda_3 t} \label{eq:1}
\end{align}
Using $\lambda_3=\lambda_1+\lambda_2$ in \eqref{eq:1}
\begin{align}
    \pr{E} &= \frac{e^{-(\lambda_1+\lambda_2)t}}{e^{-(\lambda_1+\lambda_2)t}} \times \frac{\lambda_1}{\lambda_1+\lambda_2}\\
    &= \frac{\lambda_1}{\lambda_1+\lambda_2}
\end{align}

Thus, option (a) is correct.
\end{document}

